\chapter{仓库目录结构}

本章定义了 ebuild 仓库在磁盘上的目录结构。在下文中所有说明文件或目录的地方,
指向文件或目录的符号链接同样有效,遇到这种情况,软件包管理器必须遵循操作系统的
符号链接语义并且行为不得与正常情况有所不同。

\section{顶层目录}

一个 ebuild 仓库应在磁盘上只占一个目录,其中包含以下文件和子目录:
\begin{compactitem}
\item 一个 \texttt{eapi} 文件,内容为软件包管理器在处理仓库时需要遵循的 EAPI。
    如果软件包管理器遇到了一个不支持的 EAPI,则不应对仓库执行任何操作。
\item 一个 \texttt{仓库名称}\ 文件,内容为仓库的名称,必须符合~\ref{sec:repository-names}
    节的仓库名称规范。
\item 一个 \texttt{类别}\ 文件,内容为一张仓库中类别的列表,表里的每项占一行。每个类别必须符合
    \ref{sec:category-names} 节的类别名称规范。文件中从 \texttt{\#} 字符到行尾的内容是注释,空白行忽略。
\item 每个类别一个目录,目录的名称就是类别名称。这些目录的结构如~\ref{sec:category-dirs} 节所述。
\item 一个 \texttt{系统轮廓}\ 目录,详见~\ref{sec:profiles-dir} 节。
\item 一个 \texttt{许可证}\ 目录(可选),详见~\ref{sec:licenses-dir} 节。
\item 一个 \texttt{eclass} 目录(可选),详见~\ref{sec:eclass-dir} 节。
\item 一个 \texttt{叠加层.conf} 文件(可选),详见~\ref{sec:overlay} 节。
\end{compactitem}

其他用途的文件与目录(比如骨架 ebuild 或变更日志)也可以放,但不在此规范的描述范围之内。
软件包管理器应当忽略那些无法识别的文件或目录。

\section{类别目录}
\label{sec:category-dirs}

仓库提供的每个类别都要放在一个以类别的名称命名的目录中。每个类别目录应包含:
\begin{compactitem}
\item 任意数量个软件包目录,每个目录存放一个当前类别的软件包,如~\ref{sec:package-dirs} 节所述。
    软件包目录的名称就是所存放的软件包的名称。
\end{compactitem}

类别目录中可以放其他用途的文件。为避免与软件包目录冲突,不是用于存放软件包的
目录只有名称不符合 \ref{sec:package-names} 节软件包名称规范的才可以放。

仓库提供的类别不需要每一个都有对应的目录——如果某个类别是空类别(没有软件包),那么可以没有对应的目录。

\section{软件包目录}
\label{sec:package-dirs}

一个软件包目录中包含以下几项:
\begin{compactitem}
\item 任意数量个 ebuild。ebuild 的格式与内容在第\ref{ch:ebuild-format-group}章和其他几个章节中有所描述。
\item 一个 \texttt{变更日志},格式由仓库的提供者确定,可选。
\item 一个 \texttt{files} 目录,包含 ebuild 所需的所有非源码文件,可选。
\end{compactitem}

软件包目录中基本格式的 ebuild 必须以 \texttt{<name>-<ver>.ebuild} 的形式命名,ebuild 源码包必须以
\texttt{<name>-<ver>.src.ebuild} 的形式命名,ebuild 二进制包必须以 \texttt{<name>-<ver>.<arch>.ebuild}
的形式命名;其中 \texttt{<name>} 是(非限定的)软件包名称,\texttt{<ver>} 是软件包的版本,
\texttt{<arch>} 是二进制包面向的平台,软件包管理器可以定义一个特殊的平台名称,用来表示“面向所有平台”,
如果二进制包面向的是单个平台,则 \texttt{<arch>} 必须是一个已在 \texttt{平台列表}\ 文件(见
\ref{sec:profiles-dir} 节)中列出的平台。软件包管理器必须忽视任何不按这些规则命名的 ebuild 文件。

不包含正确命名的 ebuild 的软件包目录应被视为仓库中不存在的软件包。

软件包目录中可以放其他用途的文件或目录。

\section{系统轮廓目录}
\label{sec:profiles-dir}

\texttt{系统轮廓}\ 目录应包含一个或多个储存系统轮廓的目录,这些目录会在第\ref{ch:profiles}章说明。
除此之外,\texttt{系统轮廓}\ 目录还应包含下面要说的文件和目录。如果没有特别说明,这些文件中从
\texttt{\#} 字符到行尾的内容是注释,空白行忽略。

此规范没有描述的其他文件也可以放,但软件包管理器不能依赖它们。软件包管理器应当忽略此目录中
任何无法识别的文件。

\begin{description}
\item[平台列表] 内容为一张仓库涉及平台的列表。表里的每项占一行,
    并且必须符合 \ref{sec:keyword-names} 节的平台名称规范。
\item[软件包屏蔽]
    内容为一张不带阻塞符、插槽依赖和应用标志依赖的软件包依赖说明符(详见
    \ref{sec:package-dependency-spec} 节)的列表,表里的每项占一行。任何匹配其中之一的
    软件包版本都会被视为已屏蔽,进而不会安装,除非被系统轮廓或用户配置解除屏蔽。

    此文件是可选的,与此同时,此文件可以用一个同名目录替代。替代目录下的文件,只要名称不是以点开头,
    都会按照 POSIX 语言环境下文件名的顺序进行拼接,得到的结果视为此文件的内容。而子目录则会忽略。
\item[摘要] 下文~\ref{sec:profiles.desc} 节详述。
\item[第三方镜像] 下文~\ref{sec:thirdpartymirrors} 节详述。
\item[全局应用标志] 内容为仓库中有效的全局应用标志的描述。格式在~\ref{sec:use.desc} 节详述。
    如果内容为空则此文件是可选的。
\item[局部应用标志] 内容为仓库中有效的局部应用标志的描述,以及这些标志适用的软件包。
    格式如~\ref{sec:use.desc} 节所述。如果内容为空则此文件是可选的。
\item[折叠应用标志/] 下文~\ref{sec:use.desc} 节详述。
\end{description}

\subsection{摘要文件}
\label{sec:profiles.desc}
\texttt{摘要}\ 文件罗列了仓库对外提供的系统轮廓,以及它们的状态和面向的平台。每一行的格式为:

\begin{verbatim}
<平台> <系统轮廓路径> <状态>
\end{verbatim}

其中:
\begin{compactitem}
\item \texttt{<\hspace{0em}平台\hspace{0em}>} 是系统轮廓面向的平台,必须是已在
    \texttt{平台列表}\ 中列出的值。
\item \texttt{<\hspace{0em}系统轮廓路径\hspace{0em}>} 是从 \texttt{系统轮廓}\ 目录
    到系统轮廓的(相对)路径。
\item \texttt{<\hspace{0em}状态\hspace{0em}>} 表示系统轮廓的稳定性。
    该字段的值及其含义由仓库的提供者定义。
\end{compactitem}

字段之间以空格或水平制表符分隔。

\subsection{第三方镜像文件}
\label{sec:thirdpartymirrors}
\texttt{第三方镜像}\ 文件描述了此仓库中可以通过 \texttt{mirror://} URI
有效使用的第三方镜像,以及相关下载位置的链接。每一行的格式为:
\begin{verbatim}
<第三方镜像名称> <链接 1> <链接 2> ... <链接 n>
\end{verbatim}
字段之间以空格或水平制表符分隔。当解析格式为 \texttt{mirror:/\slash 名称\slash 路径\slash
文件名}(\texttt{路径/} 部分是可选的)的 URI 时,会在 \texttt{第三方镜像}\ 文件中搜索第一个
字段是 \texttt{名称}\ 的行。然后把后续字段里的下载 URI 尾部加上 \texttt{路径\slash 文件名}\
生成尝试进行下载的 URI。

每个第三方镜像名称在文件中最多只能出现一次,当一个第三方镜像名称出现多次时软件包管理器的行为
不做定义。当一个第三方镜像根据另一个第三方镜像进行定义时软件包管理器的行为也不做定义。软件包
管理器可以选择从所列出链接的全部或一部分中尝试下载,并且可以按照不同于文件中描述的顺序进行尝试。

此文件是可选的。

\subsection{应用标志相关文件}
\label{sec:use.desc}
\texttt{全局应用标志}\ 文件的内容为此仓库中有效的全局应用标志及其描述。每一行的格式为:
\begin{verbatim}
<标志名称> - <描述>
\end{verbatim}

\texttt{局部应用标志}\ 文件的内容为有效的局部应用标志,即那些只适用于一小部分软件包,
或在不同软件包中有不同含义的标志,以及标志的描述。格式为:
\begin{verbatim}
<限定的软件包名称>:<标志名称> - <描述>
\end{verbatim}
标志针对它适用的每个软件包都必须列出一次,或者如果标志同时在 \texttt{全局应用标志}\ 和
\texttt{局部应用标志}\ 中列出,那么它必须针对每个标志含义不同于 \texttt{全局应用标志}\
所述含义的软件包都列出一次。

\texttt{折叠应用标志}\ 目录存放用于描述各种折叠应用标志的文件,如果为空,则此目录是可选的。
文件以 \texttt{<varname>.desc} 命名,其中 \texttt{<varname>} 是折叠应用标志变量(见
\ref{sec:specific-variables} 节)的小写形式。文件的内容为变量展开(见 \ref{sec:use-iuse-handling}
节)可能得到的标志以及标志的描述,文件的格式与 \texttt{全局应用标志}\ 相同。

每一个折叠应用标志变量对应的文件都是可选的。

\section{许可证目录}
\label{sec:licenses-dir}

\texttt{许可证}\ 目录应包含仓库中软件包所使用许可证的副本。文件应根据在 \texttt{LICENSE}
变量(详见~\ref{sec:optional-vars} 节)中使用的名称进行命名,并且以人类可读的形式
包含完整的许可证文本。虽不是必须但强烈推荐纯文本格式。

\section{eclass 目录}
\label{sec:eclass-dir}

\texttt{eclass} 目录应包含仓库提供的 eclass 文件,这些文件会在第\ref{ch:eclasses}章介绍。

\section{主仓库与从属仓库}
\label{sec:overlay}

\texttt{叠加层.conf} 文件用来声明仓库的从属关系。文件中从 \texttt{\#} 字符到行尾的内容是注释,
空白行忽略,正文部分的每一行是一个 \verb|键=值| 的条目。此文件中必须有一行 \verb|master=<主仓库>|,
其中 \texttt{<\hspace{0em}主仓库\hspace{0em}>} 是一个仓库名称或空字符串,
表示当前仓库从属于 \texttt{<\hspace{0em}主仓库\hspace{0em}>} 或是一个独立仓库。

当处理从属仓库时,软件包管理器会在主仓库的基础上进行“叠加”:将从属仓库视为拥有主仓库中
所有的文件和目录,如果(在相同路径下)主仓库和从属仓库有同名文件,则从属仓库的覆盖主仓库的。
这种叠加行为针对此规范定义的文件和目录,如果主仓库或从属仓库中有超出此规范范围的文件或目录,
则针对它们的行为不做定义。

仓库可以从属于从属了另一个仓库的仓库,因此仓库间的从属关系具有相对性。
在仓库的从属关系链中出现循环是不合法的,遇到这种情况软件包管理器的行为不做定义。

不从属于任何仓库的仓库称为独立仓库,从属于某个仓库的仓库称为非独立仓库。\texttt{叠加层.conf}
文件可以在独立仓库中出现,但 \texttt{<\hspace{0em}主仓库\hspace{0em}>} 必须是空字符串。

其他用途的键值也可以出现在 \texttt{叠加层.conf} 中,但软件包管理器不得依赖它们。
软件包管理器应当忽略那些意义不明的键值。

% vim: set filetype=tex fileencoding=utf8 et tw=100 spell spelllang=en :

%%% Local Variables:
%%% mode: latex
%%% TeX-master: "pms"
%%% LaTeX-indent-level: 4
%%% LaTeX-item-indent: 0
%%% TeX-brace-indent-level: 4
%%% fill-column: 100
%%% End:
