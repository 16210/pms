\chapter{eclass}
\label{ch:eclasses}

eclass 用于存储多个 ebuild 共用的代码,这大大有助于提高可维护性与减小仓库的体积。
在格式上 eclass 与 ebuild 类似,而实际上 eclass 被拉入使用它的 ebuild 成为其一部分,
因此,它们用的是同一个解释器,对可解析性有着相同的要求。

eclass 必须放在仓库顶层目录的 \texttt{eclass} 目录中——详见~\ref{sec:eclass-dir} 节。
每个 eclass 是以 \texttt{<\hspace{0em}\nolinebreak 名称\hspace{0em}\nolinebreak>.eclass}
命名的单个文件,其中 \texttt{<\hspace{0em}名称\hspace{0em}>} 就是这个 eclass 的名称,
用来被下文的 \texttt{inherit} 和 \texttt{EXPORT_FUNCTIONS} 命令引用。

\section{inherit 命令}
\label{sec:inherit}

ebuild 可以在全局区域中使用 \texttt{inherit} 命令来继承 eclass。这么做会使得 eclass
被拉入,成为 ebuild 的一部分——任何在 eclass 中定义的函数和变量都会出现在 ebuild
中,只有几个特定变量会采取特殊处理,这将在下文详述。

\texttt{inherit} 命令需要一个或多个参数,这些参数必须是 eclass 的名称(不带
路径和 \texttt{.eclass} 后缀)。参数所指代的 eclass 会按顺序依次拉入。

同一个 eclass 到最后可能会被拉入多次。

\texttt{inherit} 命令还必须确保:

\begin{compactitem}
\item 拉入 eclass 时 \texttt{ECLASS} 变量的值设为当前 eclass 的名称。
\item 一旦完成了所有继承,\texttt{INHERITED} 变量的值设为
    以空格分隔的所有用到的 eclass 的名称。
\end{compactitem}

\section{eclass 累加变量}

当 \texttt{IUSE},\texttt{REQUIRED_USE},\texttt{PROPERTIES},\texttt{RESTRICT},
\texttt{BDEPEND},\texttt{RDEPEND},\texttt{PDEPEND} 和 \texttt{IDEPEND} 变量被
eclass 赋值时会采取特殊处理。这些变量的值会在 eclass 之间累加,当前 eclass 所赋的值
会追加到上一个 eclass 拉入后,这些变量最终结果的后边。随后 eclass 所赋的值会追加到
ebuild 所赋的值的后边。如果是 \texttt{RDEPEND},这些操作要在~\ref{sec:rdepend-depend}
节所述的隐式 \texttt{RDEPEND} 规则应用之后再做。

\section{EXPORT_FUNCTIONS}

在 eclass 的环境下有一个在 ebuild 中既不能用也没有意义的命令——\texttt{EXPORT_FUNCTIONS}。
eclass 可以用它来导出 ebuild 阶段函数,使得继承它的 ebuild 得到的默认实现是 eclass
中定义的版本。\texttt{EXPORT_FUNCTIONS} 的使用通过一个例子——假想中的 \texttt{foo.eclass}
来说明,如代码清单~\ref{lst:export-functions} 所示。

\begin{listing}[h]
\caption{\texttt{EXPORT_FUNCTIONS} 示例:\texttt{foo.eclass}} \label{lst:export-functions}
\begin{verbatim}
foo_src_compile()
{
    econf --enable-gerbil \
            $(use_enable fnord)
    emake gerbil || die "Couldn't make a gerbil"
    emake || die "emake failed"
}

EXPORT_FUNCTIONS src_compile
\end{verbatim}
\end{listing}

这个例子定义了一个 eclass 版的 \texttt{src_compile} 函数并使用 \texttt{EXPORT_FUNCTIONS} 将它导出。
这样一来,此函数在所有继承了 \texttt{foo.eclass} 的 ebuild 中都将是默认实现,而如果在 ebuild 中
定义了 \texttt{src_compile},那么既可以通过调用 \texttt{foo_src_compile} 来执行 \texttt{foo.eclass}
中的函数,也可以通过调用 \texttt{default_src_compile} 来执行软件包管理器提供的默认实现。

\texttt{EXPORT_FUNCTIONS} 只能用于 ebuild 阶段函数,并且要导出的函数必须以
\texttt{eclass\hspace{0em}名称_阶段函数名称}\ 命名。

% vim: set filetype=tex fileencoding=utf8 et tw=100 spell spelllang=en :

%%% Local Variables:
%%% mode: latex
%%% TeX-master: "pms"
%%% LaTeX-indent-level: 4
%%% LaTeX-item-indent: 0
%%% TeX-brace-indent-level: 4
%%% fill-column: 100
%%% End:
