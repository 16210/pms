\chapter{名称与版本}

\section{名称限制}

名称不能是空白。软件包管理器不得给任何名称施加固定的长度上限。
软件包管理器应该根据下面这些规则说明或拒绝无效的名称。

\subsection{类别名称}
\label{sec:category-names}
类别名称可以包含 [\texttt{A-Za-z0-9+_.-}] 中的任何字符,以及 Unicode 字符集
\texttt{U+4E00-U+9FA5} 范围内的字符。不能以连字符、点号或加号开头。

\subsection{软件包名称}
\label{sec:package-names}
软件包名称可以包含 [\texttt{A-Za-z0-9+_-}] 中的任何字符,以及 Unicode 字符集
\texttt{U+4E00-U+9FA5} 范围内的字符。不能以连字符或加号开头,
并且不能以连字符或连字符加一个符合版本语法(详见~\ref{sec:version-spec} 节)的东西结尾。

\note{软件包名称不包含类别名称。术语 \textit{限定的软件包名称}\ 表示 \texttt{类别/软件包}\
这种组合。例如 \texttt{app-editors/vim} 就是一个限定的软件包名称。}

\subsection{插槽名称}
\label{sec:slot-names}
插槽名称可以包含 [\texttt{A-Za-z0-9+_.-}] 中的任何字符,以及 Unicode 字符集
\texttt{U+4E00-U+9FA5} 范围内的字符。不能以连字符、点号或加号开头。

\subsection{应用标志名称}
应用标志名称可以包含 [\texttt{A-Za-z0-9+-}] 中的任何字符,以及 Unicode 字符集
\texttt{U+4E00-U+9FA5} 范围内的字符。不能以连字符或加号开头。
如果是带前缀的折叠应用标志,根据 \ref{sec:use-iuse-handling}
节所述的展开方式名称中还会带有下划线。

\subsection{仓库名称}
\label{sec:repository-names}
仓库名称可以包含 [\texttt{A-Za-z0-9_-}] 中的任何字符,以及 Unicode 字符集
\texttt{U+4E00-U+9FA5} 范围内的字符。不能以连字符开头。
另外,每一个仓库名称必须也是一个有效的软件包名称。

\subsection{许可证名称}
\label{sec:license-names}
许可证名称可以包含 [\texttt{A-Za-z0-9+_.-}] 中的任何字符,以及 Unicode 字符集
\texttt{U+4E00-U+9FA5} 范围内的字符。不能以连字符、点号或加号开头。

\subsection{平台名称}
\label{sec:keyword-names}
平台名称可以包含 [\texttt{A-Za-z0-9_-}] 中的任何字符,以及 Unicode 字符集
\texttt{U+4E00-U+9FA5} 范围内的字符。不能以连字符开头。

\subsection{EAPI 值}
\label{sec:eapi-value}
EAPI 值可以包含 [\texttt{A-Za-z0-9+_.-}] 中的任何字符,以及 Unicode 字符集
\texttt{U+4E00-U+9FA5} 范围内的字符。不能以连字符、点号或加号开头。

\subsection{系统轮廓名称}
\label{sec:profile-names}
系统轮廓名称可以包含 [\texttt{A-Za-z0-9_.-}] 中的任何字符,以及 Unicode 字符集
\texttt{U+4E00-U+9FA5} 范围内的字符。不能以连字符、点号、下划线开头或结尾。

\subsection{软件包分组名称}
\label{sec:group-names}
软件包分组名称可以包含 [\texttt{A-Za-z0-9_-}] 中的任何字符,以及 Unicode 字符集
\texttt{U+4E00-U+9FA5} 范围内的字符。不能以连字符开头或结尾。

\section{版本规范}
\label{sec:version-spec}
软件包版本可以分为若干个组成部分,从左到右分别是:
\begin{compactitem}
\item 一个可选的纪元部分,格式为一个无符号整数加一个下划线。
\item 一个或多个数字部分,首个数字部分的格式为一个无符号整数,非首个数字部分的格式为
    一个点号加一个无符号整数。
\item 一个可选的字母部分,格式为一个 [\texttt{a-z}] 中的字符(小写字母)。
\item 一个可选的后缀部分,格式为一个 \texttt{_alpha},\texttt{_beta},\texttt{_pre},
    \texttt{_rc} 或 \texttt{_p} 字符串,加一个可选的无符号整数。
\item 一个可选的修订部分,格式为字符串 \texttt{-r} 加一个无符号整数,这个整数称为“修订号”。
\end{compactitem}

这些组成部分中的无符号整数允许带有前导 0。

\section{版本比较}

版本的比较是按照算法~\ref{alg:version-comparison} 及其子算法,从左到右逐个组成部分进行的。
如果子算法能比出高低,则它就是整个版本比较的结果;
如果子算法不能比出高低,那么比较过程就从调用它的地方继续进行。

\begin{algorithm}[h]
\caption{版本比较顶层逻辑} \label{alg:version-comparison}
\begin{algorithmic}[1]
    \STATE 令 $A$ 和 $B$ 等于要比较的两个版本
    \STATE 调用算法~\ref{alg:version-comparison-epoch} 比较它们的纪元部分
    \STATE 调用算法~\ref{alg:version-comparison-numeric} 比较它们的数字部分
    \IF{$A$ 和 $B$ 都带有字母部分}
        \STATE 调用算法~\ref{alg:version-comparison-letter} 比较它们的字母部分
    \ELSIF{$A$ 带有字母部分}
        \RETURN $A<B$
    \ELSIF{$B$ 带有字母部分}
        \RETURN $A>B$
    \ENDIF
    \IF{$A$ 和 $B$ 都带有后缀部分}
        \STATE 调用算法~\ref{alg:version-comparison-suffix} 比较它们的后缀部分
    \ELSIF{$A$ 带有后缀部分}
        \RETURN $A<B$
    \ELSIF{$B$ 带有后缀部分}
        \RETURN $A>B$
    \ENDIF
    \STATE 调用算法~\ref{alg:version-comparison-revision} 比较它们的修订部分
    \RETURN $A=B$
\end{algorithmic}
\end{algorithm}

\begin{algorithm}[h]
\caption{版本纪元部分的比较逻辑} \label{alg:version-comparison-epoch}
\begin{algorithmic}[1]
    \STATE 令 $Ae$ 等于 $A$ 的纪元部分去掉下划线,如果纪元部分省略则等于 $\texttt{0}$
    \STATE 令 $Be$ 等于 $B$ 的纪元部分去掉下划线,如果纪元部分省略则等于 $\texttt{0}$
    \IF{以整数进行比较 $Ae>Be$}
        \RETURN $A>B$
    \ELSIF{以整数进行比较 $Ae<Be$}
        \RETURN $A<B$
    \ENDIF
\end{algorithmic}
\end{algorithm}

\begin{algorithm}[p]
\caption{版本数字部分的比较逻辑} \label{alg:version-comparison-numeric}
\begin{algorithmic}[1]
    \STATE 记 $An_k$ 和 $Bn_k$ 分别为 $A$ 和 $B$ 的第 $k$ 个 $0$ 索引的数字部分
    \IF{以整数进行比较 $An_0>Bn_0$}
        \RETURN $A>B$
    \ELSIF{以整数进行比较 $An_0<Bn_0$}
        \RETURN $A<B$
    \ENDIF
    \STATE 令 $Ann$ 等于 $A$ 的数字部分的个数
    \STATE 令 $Bnn$ 等于 $B$ 的数字部分的个数
    \FORALL{满足 $i\geq1$ 且 $i<Ann$ 且 $i<Bnn$ 的 $i$,按升序依次}
        \STATE 令 $An'_i$ 等于 $An_i$ 去掉点号
        \STATE 令 $Bn'_i$ 等于 $Bn_i$ 去掉点号
        \IF{以整数进行比较 $An'_i>Bn'_i$}
            \RETURN $A>B$
        \ELSIF{以整数进行比较 $An'_i<Bn'_i$}
            \RETURN $A<B$
        \ENDIF
    \ENDFOR
    \IF{$Ann>Bnn$}
        \RETURN $A>B$
    \ELSIF{$Ann<Bnn$}
        \RETURN $A<B$
    \ENDIF
\end{algorithmic}
\end{algorithm}

\begin{algorithm}[p]
\caption{版本字母部分的比较逻辑} \label{alg:version-comparison-letter}
\begin{algorithmic}[1]
    \STATE 令 $Al$ 和 $Bl$ 分别等于 $A$ 和 $B$ 的字母部分
    \IF{以 Unicode 字符串进行比较 $Al>Bl$}
        \RETURN $A>B$
    \ELSIF{以 Unicode 字符串进行比较 $Al<Bl$}
        \RETURN $A<B$
    \ENDIF
\end{algorithmic}
\end{algorithm}

\begin{algorithm}[t]
\caption{版本后缀部分的比较逻辑} \label{alg:version-comparison-suffix}
\begin{algorithmic}[1]
    \STATE 令 $As$ 和 $Bs$ 分别等于 $A$ 和 $B$ 的后缀部分
    \IF{$As$ 和 $Bs$ 是相同的类型(比如都是 \texttt{_alpha} 或 \texttt{_beta})}
        \STATE 令 $As'$ 等于 $As$ 的整数部分,如果省略则等于 \texttt{0}
        \STATE 令 $Bs'$ 等于 $Bs$ 的整数部分,如果省略则等于 \texttt{0}
        \IF{以整数进行比较 $As'>Bs'$}
            \RETURN $A>B$
        \ELSIF{以整数进行比较 $As'<Bs'$}
            \RETURN $A<B$
        \ENDIF
    \ELSIF{以 $\mbox{\texttt{_alpha}}<\mbox{\texttt{_beta}}<\mbox{\texttt{_pre}}<\mbox{\texttt{_rc}}<\mbox{\texttt{_p}}$
        的顺序进行比较 $As$ 的类型大于 $Bs$ 的类型}
        \RETURN $A>B$
    \ELSE
        \RETURN $A<B$
    \ENDIF
\end{algorithmic}
\end{algorithm}

\begin{algorithm}[t]
\caption{版本修订部分的比较逻辑} \label{alg:version-comparison-revision}
\begin{algorithmic}[1]
    \STATE 令 $Ar$ 等于 $A$ 的修订部分中的修订号,如果修订部分省略则等于 $\texttt{0}$
    \STATE 令 $Br$ 等于 $B$ 的修订部分中的修订号,如果修订部分省略则等于 $\texttt{0}$
    \IF{以整数进行比较 $Ar>Br$}
        \RETURN $A>B$
    \ELSIF{以整数进行比较 $Ar<Br$}
        \RETURN $A<B$
    \ENDIF
\end{algorithmic}
\end{algorithm}

\section{版本唯一性}

在一个给定的仓库中,任意两个软件包不能具有相同的限定软件包名与相等的版本。
例如,\texttt{foo-bar/baz-1.0.2},\texttt{foo-bar/baz-1.0.2-r0} 和 \texttt{foo-bar/baz-0_1.0.2}
在一个仓库中最多只能有一个。

% vim: set filetype=tex fileencoding=utf8 et tw=100 spell spelllang=en :

%%% Local Variables:
%%% mode: latex
%%% TeX-master: "pms"
%%% LaTeX-indent-level: 4
%%% LaTeX-item-indent: 0
%%% TeX-brace-indent-level: 4
%%% fill-column: 100
%%% End:
