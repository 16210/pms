\chapter{ebuild 文件格式与软件包分组}
\label{ch:ebuild-format-group}
\section{ebuild 文件格式}
\label{sec:ebuild-format}

ebuild 文件格式的基本形式是 bash 脚本格式的子集。解释器假定是 GNU bash,版本为
5.0 或更高的版本。如果可能的话,软件包管理器应该把 shell 的兼容等级设为此版本,
同时必须确保此类兼容性设置(比如 \texttt{BASH_COMPAT} 变量)不会导出到外部程序中。

bash 的 \texttt{failglob} 选项会在 ebuild 拉入前设置。因此
在 ebuild 拉入过程中文件名通配符展开时的模式匹配失败会造成一个错误。

名称引用变量(bash 4.3 版本引入)不得使用,除非是在局部区域中。

文件的编码必须是带有 Unix 风格换行符的 UTF-8。被拉入时,ebuild 必须定义某些变量
和函数(具体内容详见第\ref{ch:ebuild-vars}章和第\ref{ch:ebuild-functions}章),
不得以任何方式调用外部程序,往标准输出或标准错误中写入,或者修改系统状态。

\section{扩展 ebuild 格式}
在 ebuild 基本格式的基础上,软件包有源码包和二进制包两种扩展格式。

\subsection{源码包}
ebuild 源码包是 tar 格式的归档,其中包含以下文件和目录:
\begin{compactitem}
\item 软件包基本格式的 ebuild 文件,命名规则和 \ref{sec:package-dirs} 节中的相同。
\item \texttt{FILESDIR} 目录,含有软件包所需的所有非源码文件。
\item \texttt{DISTDIR} 目录,含有 \texttt{SRC_URI} 变量中所有出现过的 URI(不考虑应用标志条件)
    下载得到的文件。如果 URI 后面使用了箭头,那么保存到此目录中的文件名应改为箭头右侧的文件名。
\end{compactitem}

归档中文件名的字符集必须是 Unicode,编码由实现规定。

\subsection{二进制包}
ebuild 二进制包是 tar 格式的归档,其中包含以下文件和目录:
\begin{compactitem}
\item 软件包基本格式的 ebuild 文件,命名规则和 \ref{sec:package-dirs} 节中的相同。
\item \texttt{D} 目录,含有软件包将要合并的所有文件。
\item \texttt{环境.sh} 文件,用于构建和安装过程之间的环境保存,详见 \ref{sec:ebuild-env-state} 节。
\end{compactitem}

归档中文件名的字符集和编码必须和源码包保持一致。

\section{软件包分组}
软件包管理器应当允许用户创建和管理自定义的软件包分组,同时维护以下分组:
\begin{description}
\item[所有] 当前系统上安装的所有 ebuild 软件包。
\item[显式安装] 在软件包管理器提供的安装命令(或界面,如果提供的话)中指定的软件包。
\item[隐式安装] \texttt{所有}\ 分组中不属于 \texttt{显式安装}\ 分组的软件包。
\end{description}

这 3 个分组由软件包管理器自动维护,是否允许用户干预不做规定。

% vim: set filetype=tex fileencoding=utf8 et tw=100 spell spelllang=en :

%%% Local Variables:
%%% mode: latex
%%% TeX-master: "pms"
%%% LaTeX-indent-level: 4
%%% LaTeX-item-indent: 0
%%% TeX-brace-indent-level: 4
%%% fill-column: 100
%%% End:
