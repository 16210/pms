\chapter{软件包依赖}
\label{ch:dependencies}

\section{依赖的种类}
\label{sec:dependency-classes}

ebuild 之间有四种依赖关系:

\begin{compactitem}
\item 构建依赖(\texttt{BDEPEND}),软件包的构建过程所需要的软件包。这种依赖必须在 \texttt{pkg_setup}
    阶段函数执行前安装,并且在所有 \texttt{src_*} 阶段函数执行期间确保可用。
    如果正在安装的是二进制包,则忽略构建依赖。
\item 安装依赖(\texttt{IDEPEND}),软件包的 \texttt{pkg_preinst} 和 \texttt{pkg_postinst}
    阶段函数所需要的软件包。ebuild 除了在 \texttt{pkg_preinst} 和 \texttt{pkg_postinst} 中调用它们之外,
    也可以在 \texttt{pkg_prerm} 和 \texttt{pkg_postrm} 中调用它们,但不得依赖它们此时还能使用。
\item 强运行依赖(\texttt{RDEPEND}),软件包运行所必须的软件包。这种依赖要在 ebuild 安装的结果
    被视为可用之前安装并确保可用。
\item 弱运行依赖(\texttt{PDEPEND}),也叫“后置依赖”。软件包的次要功能需要或在某些不常见的情况下会用到,
    但不装也不会影响主要功能的软件包。这种依赖可以在 ebuild 安装之后的某个时间安装。
\end{compactitem}

各个阶段函数执行前需要安装的依赖种类在表 \ref{tab:phase-function-dependency-classes} 中列出。

\begin{centertable}{针对不同阶段函数需要满足的依赖种类}
    \label{tab:phase-function-dependency-classes}
    \begin{tabular}{P{13.5em} P{0.6\textwidth}}
      \toprule
      \multicolumn{1}{c}{\textbf{阶段函数}} &
      \multicolumn{1}{c}{\textbf{需要满足的依赖种类}} \\
      \midrule
      \texttt{pkg_pretend},\texttt{pkg_info},\texttt{pkg_nofetch} &
          无(可以依赖 \texttt{最小安装}\ 分组里的软件包) \\
      \addlinespace
      \texttt{pkg_setup} & 如果是在源码构建过程中执行,就和 \texttt{src_unpack}
          一样,否则就和 \texttt{pkg_pretend} 一样。 \\
      \addlinespace
      \texttt{src_unpack},\texttt{src_prepare},\texttt{src_configure},\texttt{src_compile},\texttt{src_test},
          \texttt{src_install} & \texttt{BDEPEND} \\
      \addlinespace
      \texttt{pkg_preinst},\texttt{pkg_postinst} &
          \texttt{IDEPEND} \\
      \addlinespace
      \texttt{pkg_prerm},\texttt{pkg_postrm},\texttt{pkg_config} &
          \texttt{RDEPEND} \\
      \bottomrule
    \end{tabular}
\end{centertable}

\section{依赖说明符格式}
\label{sec:dependency-spec}

本节描述的说明符不仅是在 4 个依赖变量中使用的依赖说明符,同时也是在
\texttt{SRC_URI}、\texttt{LICENSE}、\texttt{REQUIRED_USE}、\texttt{PROPERTIES} 和
\texttt{RESTRICT} 中使用的“依赖风格的说明符”。

说明符可以是以下几种形式:

\begin{compactitem}
\item 软件包依赖说明符。允许出现在 \texttt{BDEPEND}、\texttt{RDEPEND}、
    \texttt{PDEPEND}、\texttt{IDEPEND} 中。
\item 形如 \texttt{协议://主机/路径}\ 的 URI,再加可选的箭头语法。允许出现在 \texttt{SRC_URI} 中。
\item 一个单独的文件名,允许出现在 \texttt{SRC_URI} 中。
\item 许可证名称(比如 \texttt{GPL-2})。允许出现在 \texttt{LICENSE} 中。
\item 应用标志名称,前面可选地带一个感叹号。允许出现在 \texttt{REQUIRED_USE} 中。
\item 属性标记。允许出现在 \texttt{PROPERTIES} 中。
\item 行为约束标记。允许出现在 \texttt{RESTRICT} 中。
\item 一个全选组,格式为:\texttt{'('(说明符\ 空格)* 说明符')'},即,一对圆括号,
    括号里是一个或多个以空格分隔的说明符。允许出现在 \texttt{BDEPEND}、
    \texttt{RDEPEND}、\texttt{PDEPEND}、\texttt{IDEPEND}、\texttt{SRC_URI}、\texttt{LICENSE}、
    \texttt{REQUIRED_USE}、\texttt{PROPERTIES}、\texttt{RESTRICT} 中。
\item 一个任选组,格式为:\texttt{'||('(说明符\ 空格)* 说明符')'},即,一个字符串 \texttt{||},
    一对圆括号,括号里是一个或多个以空格分隔的说明符。允许出现在 \texttt{BDEPEND}、
    \texttt{RDEPEND}、\texttt{PDEPEND}、\texttt{IDEPEND}、\texttt{LICENSE}、\texttt{REQUIRED_USE} 中。
\item 一个单选组,格式与任选组相比除了开头的字符串换成了
    \texttt{\textasciicircum\textasciicircum} 外,其他的都一样。允许出现在 \texttt{REQUIRED_USE} 中。
\item 一个至多单选组,格式与任选组相比除了开头的字符串换成了
    \texttt{??} 外,其他的都一样。允许出现在 \texttt{REQUIRED_USE} 中。
\item 一个应用标志条件组,格式为:\texttt{'!'?\ 标志名称\ '?('(说明符\ 空格)* 说明符')'},即,
    一个可选的感叹号,一个应用标志名称,一个问号,一对圆括号,括号里是一个或多个以空格分隔的说明符。
    允许出现在 \texttt{BDEPEND}、\texttt{RDEPEND}、\texttt{PDEPEND}、\texttt{IDEPEND}、\texttt{SRC_URI}、
    \texttt{LICENSE}、\texttt{REQUIRED_USE}、\texttt{PROPERTIES}、\texttt{RESTRICT} 中。
\end{compactitem}

\subsection{全选说明符}

在全选组中,所有的说明符都生效(在 \texttt{SRC_URI},\texttt{LICENSE},\texttt{PROPERTIES},
\texttt{RESTRICT} 中)或必须匹配(在 \texttt{BDEPEND},\texttt{RDEPEND},\texttt{PDEPEND},
\texttt{IDEPEND},\texttt{REQUIRED_USE} 中)。

\subsection{应用标志条件说明符}

在应用标志条件组中,如果相关的应用标志被启用(若有感叹号前缀则为禁用),则所有的说明符都
生效(在 \texttt{SRC_URI},\texttt{LICENSE},\texttt{PROPERTIES},
\texttt{RESTRICT} 中)或必须匹配(在 \texttt{BDEPEND},\texttt{RDEPEND},\texttt{PDEPEND},
\texttt{IDEPEND},\texttt{REQUIRED_USE} 中)。否则忽略这个组。

在应用标志条件组中出现 \texttt{IUSE_EFFECTIVE} 中不存在的应用标志会导致软件包被视为已屏蔽。

\subsection{任选说明符}

在任选组中,至少有一个说明符生效(在 \texttt{LICENSE} 中)或必须匹配(在 \texttt{BDEPEND},
\texttt{RDEPEND},\texttt{PDEPEND},\texttt{IDEPEND},\texttt{REQUIRED_USE} 中)。

\subsection{单选说明符}

在单选组中必须有且仅有一个说明符匹配。

\subsection{至多单选说明符}

在至多单选组中,最多只能有一个说明符匹配。

\section{软件包依赖说明符}
\label{sec:package-dependency-spec}

软件包依赖说明符的格式如下:

\begin{verbatim}
<阻塞符><操作符><限定的软件包名称><连字符加版本><插槽依赖><应用标志依赖>
\end{verbatim}

其中操作符和连字符加版本必须一起出现,除限定的软件包名称外,其他所有部分都是可选的。
软件包管理器遇到不合规的说明符时应当警告或报错。

\subsection{操作符}
\label{sec:dep-operator}

在软件包依赖中可以使用以下几种操作符:

\begin{description}
\item[\texttt{<}] 严格小于指定的版本。
\item[\texttt{<=}] 小于或等于指定的版本。
\item[\texttt{=}] 与指定的版本完全相等。特殊例外:如果指定的版本后面紧跟着一个星号,
    那么只比较给定数量的版本组成部分,即,星号扮演着后面组成部分通配符的角色。
    要使用星号,必须确保说明符去掉星号后仍然是有效的。(星号与任何其他操作符一起使用都是非法的)
\item[\texttt{\textasciitilde}] 在忽略修订部分的情况下与指定的版本完全相等。
\item[\texttt{>=}] 大于或等于指定的版本。
\item[\texttt{>}] 严格大于指定的版本。
\end{description}

\subsection{阻塞符}

阻塞符是一个感叹号,如果软件包依赖说明符中加了阻塞符,那么指定的软件包没有安装才算作依赖满足。

阻塞符不能出现在 \texttt{PDEPEND} 中。

\subsection{插槽依赖}
\label{sec:slot-dep}

插槽依赖由一个冒号和以下几种形式之一组成:
\begin{description}
\item[=] 表示任何插槽都可以接受。对于运行依赖,在软件包安装完成后,如果依赖升级或降级到一个
    主插槽和子插槽不全相同的版本,软件包需要重新构建。
\item[插槽] 表示只有给定的主插槽可以接受。对于运行依赖,在软件包安装完成后,如果依赖升级
    或降级到一个子插槽不同的版本,软件包不需要重新构建。
\item[插槽\hspace{0em}=] 表示只有给定的主插槽可以接受。对于运行依赖,在软件包安装完成后,
    如果依赖升级或降级到一个子插槽不同的版本,软件包需要重新构建。
\item[插槽/子插槽] 表示只有给定的主插槽和子插槽可以接受。
\end{description}

如果没有插槽依赖,表示任何插槽都可以接受,对于运行依赖,在软件包安装完成后,
如果依赖升级或降级到一个插槽不同的版本,软件包不需要重新构建。

在 \texttt{PDEPEND} 和任选组中使用带等号的插槽依赖是无效的。

\subsection{应用标志依赖}
\label{sec:use-dep}

应用标志依赖由以下几种形式之一组成:
\begin{description}
\item[{[标志]}] 被依赖的软件包的此标志必须启用。
\item[{[-标志]}] 被依赖的软件包的此标志必须禁用。
\item[{[标志\hspace{0em}=]}] 如果启用了此标志,则被依赖的软件包的此标志必须启用,否则必须禁用。
\item[{[!\hspace{0em}标志\hspace{0em}=]}] 如果启用了此标志,则被依赖的软件包的此标志必须禁用,否则必须启用。
\item[{[标志?]}] 如果启用了此标志,则被依赖的软件包的此标志必须启用。
\item[{[!\hspace{0em}标志?]}] 如果禁用了此标志,则被依赖的软件包的此标志必须禁用。
\end{description}

在这 6 种形式的基础上,标志名称的后面可以紧跟一个
由 \texttt{(+)} 或 \texttt{(-)} 表示的“默认值”。前者表示,当应用标志依赖应用于一个
\texttt{IUSE_EFFECTIVE} 中没有相关标志的软件包时,软件包管理器的行为应等同于
标志存在并且已启用;而后者则是存在并且已禁用。
如果没有指定这种默认值,将应用标志依赖应用于 \texttt{IUSE_EFFECTIVE} 中
没有相关标志的 ebuild 将会产生一个错误。

ebuild 在自己的 \texttt{IUSE_EFFECTIVE} 中没有相关标志的情况下使用后 4 种形式的
应用标志依赖会导致自身被软件包管理器屏蔽。

多个依赖可以用逗号分隔写在一起,例如 \texttt{[第一个,-第二个,\hspace{0em}第三个?]}。

如果指定了多个依赖,那么每一个都必须满足。

% vim: set filetype=tex fileencoding=utf8 et tw=100 spell spelllang=en :

%%% Local Variables:
%%% mode: latex
%%% TeX-master: "pms"
%%% LaTeX-indent-level: 4
%%% LaTeX-item-indent: 0
%%% TeX-brace-indent-level: 4
%%% fill-column: 100
%%% End:
