\chapter{EAPI}

\section{定义}
\label{sec:defined-eapi}
EAPI 可以理解为 ebuild API,也就是软件包所遵循的这套规范的“版本”,
它的值是一个~\ref{sec:eapi-value} 节所规定的字符串。

如果软件包管理器遇到了一个带有无法识别的 EAPI 的软件包版本,
则不应尝试对它进行任何操作。比如,可以完全忽略该软件包版本,
或者将该软件包版本标记为已屏蔽。

软件包管理器不应对 EAPI 执行除相等之外的任何比较操作。

此规范定义的是 EAPI ‘\TheCurrentEAPI ’。
\ChangeWhenAddingAnEAPI{1}

\section{保留的 EAPI}

\begin{compactitem}
\item 保留仅由一个整数所组成的 EAPI 用作此规范将来的版本。
\item 保留以字符串 \texttt{gentoo-} 开头的 EAPI 用作
    Gentoo Linux 软件包仓库兼容的实验用途。
\end{compactitem}

% vim: set filetype=tex fileencoding=utf8 et tw=100 spell spelllang=en :

%%% Local Variables:
%%% mode: latex
%%% TeX-master: "pms"
%%% LaTeX-indent-level: 4
%%% LaTeX-item-indent: 0
%%% TeX-brace-indent-level: 4
%%% fill-column: 100
%%% End:
